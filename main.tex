
\documentclass[submit]{ipsj}
%\documentclass{ipsj}


\usepackage{latexsym}
\usepackage{booktabs}
\usepackage{url}
\usepackage{nidanfloat}
\usepackage{afterpage}
\usepackage{setspace}
\usepackage{multirow}
\usepackage{here}
\usepackage{amsmath,amssymb}
\usepackage{bm}
%\usepackage{graphicx}
\usepackage[dvipdfmx]{graphicx}
\usepackage{subcaption}
\captionsetup{compatibility=false}
\usepackage{verbatim}
\usepackage{wrapfig}
\usepackage{ascmac}
%\bibliographystyle{unsrt}
\usepackage{algorithm}
\usepackage{algorithmic}
%

\def\Underline{\setbox0\hbox\bgroup\let\\\endUnderline}
\def\endUnderline{\vphantom{y}\egroup\smash{\underline{\box0}}\\}
\def\|{\verb|}

%\setcounter{巻数}{59}
%\setcounter{号数}{1}
%\setcounter{page}{1}


\受付{2018}{1}{31}
%\再受付{2015}{7}{16}   %省略可能
%\再再受付{2015}{7}{20} %省略可能
%\再再受付{2015}{11}{20} %省略可能
\採録{2018}{2}{27}




\begin{document}


\title{画像付きフェイクニュースとジョークニュースの\\検出・分類に向けた機械学習モデルの検討}

%\etitle{How to Prepare Your Paper for IPSJ Journal \\
%(ipsj.cls version 2.01)}

\affiliate{UEC}{電気通信大学\\
UEC, Chofu, Tokyo 182--8585, Japan}

\author{柳 裕太}{Yuta Yanagi}{UEC}[yanagi.yuta@ohsuga.lab.uec.ac.jp]
\author{田原 康之}{Yasuyuki Tahara}{UEC}[ohsuga@uec.ac.jp]
\author{大須賀 昭彦}{Akihiko Ohsuga}{UEC}[tahara@uec.ac.jp]
\author{清 雄一}{Yuichi Sei}{UEC}[seiuny@uec.ac.jp]

\begin{abstract}
    %背景
    SNSの発展によりあらゆる情報入手が容易になった反面、故意に作成された虚偽の情報であるフェイクニュースが社会問題になっている。
    特に画像と併せて発信されたものは、テキストのみならず画像と併せた分析アプローチが有効である。

    %既存課題
    既にテキスト・画像をCNNによって分析して真偽を判定する自動判別モデルが提案されているものの、
    ジョークとしての嘘情報と人を欺くための嘘情報が区別されていない。

    %提案
    本研究では、正しい情報・ジョーク情報・人を欺くための情報の3カテゴリを分類することで、より画像つきフェイクニュースの検出精度を向上させることを目指した.


    %実験結果
    実験結果、3カテゴリでもマクロF値が0.93と良好な結果を示した。
\end{abstract}
%

\begin{jkeyword}
機械学習,フェイクニュース,ソーシャルネットワーク,Text-CNN
\end{jkeyword}

\maketitle

\input{chapters/introduction}
\input{chapters/related_reseaches}
\section{研究目的}
\label{ch:purpose}
%
今回対象とするカテゴリの投稿例を今回扱ったデータセットから抜粋したものが以下の図\ref{fig:examples}である.

\begin{figure*}[ht]
    \centering
    \begin{subfigure}[b]{0.4\textwidth}
        \includegraphics[height=6.5cm]{images/real_example_boston.jpg}
        \caption{Boston RIC released this flier showing at large suspect Dzhokhar Tsarnaev. He may be armed \& dangerous}
        \label{fig:real}
    \end{subfigure}
    \hfill % separation between the subfigures
    \begin{subfigure}[b]{0.57\textwidth}
        \includegraphics[height=6.5cm]{images/humor_example_boston.jpg}
        \caption{The detail of these photos used to identify the Boston Marathon bombing suspect is bananas…}
        \label{fig:humor}
    \end{subfigure}
    \bigskip 
    \centering
    \begin{subfigure}[b]{\textwidth}
        \includegraphics[width=\textwidth]{images/fake_example_boston.jpg}
        \caption{Reddit is on to something... Boston Bomber \#2 sure looks like missing student Sunil Tripathi. }
        \label{fig:fake}
    \end{subfigure}
    \caption{当研究で扱う3カテゴリの投稿例: (a)正しいニュース,(b)ジョークニュース,(c)フェイクニュース}
    \label{fig:examples}
\end{figure*}

いずれも2013年に発生したボストンマラソン爆弾テロ事件に関してTwitter上で投稿されたものであった.
図\ref{fig:real}は実際にボストン市傘下が作成した被疑者の情報をChicago Sun-TimesがTwitterに投稿したもの,
図\ref{fig:humor}はテロ後にRedditや4chanにて有志が実行犯の調査が行われた件に対して
``bananas''と茶化すような言葉を投げかけているもの,
図\ref{fig:fake}は実際に上記掲示板上で実行犯の調査が行われた結果,
全くの別人を槍玉に挙げているものである.

実際に,ボストンマラソン後ではインターネット上で盛んに犯人探しが行われた結果,
事件前に行方不明になっていたスニル・トリパティ(Sunil Tripathi)さんが犯人として扱われ,
更にその後一般報道メディアもトリパティさんの家族に取材が行われるなど,
フェイクニュースが実害として現実になった\cite{gray_2013}.
これを受け,Redditでは実際に犯人探しの過熱で無関係の個人とその家族に迷惑をかけたとして謝罪した\cite{laird_2013}.

%
当研究では,上記対象を正確に3カテゴリへ分類するモデルを構築することを目標としている.
具体的には,入力として画像と文章を持ち,それに対してどのカテゴリが該当するかを出力するモデルとなる.

当研究を更に発展させると,SNS上でユーザや運営側を支援するエージェント開発に繋げることができる.
%
\input{chapters/methology}
\input{chapters/experiment}
\input{chapters/evaluation}
\input{chapters/conclusion}

%謝辞
\input{chapters/thanks}

\begin{thebibliography}{10}

\bibitem{10.1257/jep.31.2.211}
Hunt Allcott and Matthew Gentzkow.
 Social media and fake news in the 2016 election.
 {\em Journal of Economic Perspectives}, 31(2):211--36, 5 2017.

\bibitem{Granik8100379}
M.~Granik and V.~Mesyura.
 Fake news detection using naive bayes classifier.
 In {\em 2017 IEEE First Ukraine Conference on Electrical and Computer
  Engineering (UKRCON)}, pages 900--903, 5 2017.

\bibitem{Gilda8305411}
S.~Gilda.
 Evaluating machine learning algorithms for fake news detection.
 In {\em 2017 IEEE 15th Student Conference on Research and Development
  (SCOReD)}, pages 110--115, 12 2017.

\bibitem{松尾省吾2018master}
松尾 省吾.
 機械学習を用いた流言の検出に関する研究.
 Master's thesis, 電気通信大学院, 2018.

\bibitem{Wu:2018:TFF:3159652.3159677}
Liang Wu and Huan Liu.
 Tracing fake-news footprints: Characterizing social media messages by
  how they propagate.
 In {\em Proceedings of the Eleventh ACM International Conference on
  Web Search and Data Mining}, WSDM '18, pages 637--645, New York, NY, USA,
  2018. ACM.

\bibitem{W16-0802}
Victoria Rubin, Niall Conroy, Yimin Chen, and Sarah Cornwell.
 Fake news or truth? using satirical cues to detect potentially
  misleading news.
 In {\em Proceedings of the Second Workshop on Computational
  Approaches to Deception Detection}, pages 7--17. Association for
  Computational Linguistics, 2016.

\bibitem{DBLP:journals/corr/HorneA17}
Benjamin~D. Horne and Sibel Adali.
 This just in: Fake news packs a lot in title, uses simpler,
  repetitive content in text body, more similar to satire than real news.
 {\em CoRR}, abs/1703.09398, 2017.

\bibitem{Jin:2017:MFR:3123266.3123454}
Zhiwei Jin, Juan Cao, Han Guo, Yongdong Zhang, and Jiebo Luo.
 Multimodal fusion with recurrent neural networks for rumor detection
  on microblogs.
 In {\em Proceedings of the 25th ACM International Conference on
  Multimedia}, MM '17, pages 795--816, New York, NY, USA, 2017. ACM.

\bibitem{Wang:2018:EEA:3219819.3219903}
Yaqing Wang, Fenglong Ma, Zhiwei Jin, Ye~Yuan, Guangxu Xun, Kishlay Jha, Lu~Su,
  and Jing Gao.
 Eann: Event adversarial neural networks for multi-modal fake news
  detection.
 In {\em Proceedings of the 24th ACM SIGKDD International Conference
  on Knowledge Discovery \&\#38; Data Mining}, KDD '18, pages 849--857, New
  York, NY, USA, 2018. ACM.

\bibitem{DBLP:journals/corr/SimonyanZ14a}
Karen Simonyan and Andrew Zisserman.
 Very deep convolutional networks for large-scale image recognition.
 {\em CoRR}, abs/1409.1556, 2014.

\bibitem{DBLP:journals/corr/Kim14f}
Yoon Kim.
 Convolutional neural networks for sentence classification.
 {\em CoRR}, abs/1408.5882, 2014.

\bibitem{7298935}
O.~Vinyals, A.~Toshev, S.~Bengio, and D.~Erhan.
 Show and tell: A neural image caption generator.
 In {\em 2015 IEEE Conference on Computer Vision and Pattern
  Recognition (CVPR)}, pages 3156--3164, 6 2015.

\bibitem{7410636}
S.~Antol, A.~Agrawal, J.~Lu, M.~Mitchell, D.~Batra, C.~L. Zitnick, and
  D.~Parikh.
 Vqa: Visual question answering.
 In {\em 2015 IEEE International Conference on Computer Vision
  (ICCV)}, pages 2425--2433, 12 2015.

\bibitem{holan_2018}
Angie~Drobnic Holan.
 The principles of the truth-o-meter: How we fact-check, 2 2018.

\bibitem{tamir}
Mike Tamir.
 About fakerfact.

\bibitem{harmanci_2012}
Reyhan Harmanci.
 11 viral photos that aren't hurricane sandy, 10 2012.

\bibitem{collobert2011natural}
Ronan Collobert, Jason Weston, L{\'e}on Bottou, Michael Karlen, Koray
  Kavukcuoglu, and Pavel Kuksa.
 Natural language processing (almost) from scratch.
 {\em Journal of Machine Learning Research}, 12(Aug):2493--2537, 2011.

\bibitem{KalchbrennerACL2014}
Nal Kalchbrenner, Edward Grefenstette, and Phil Blunsom.
 A convolutional neural network for modelling sentences.
 {\em Proceedings of the 52nd Annual Meeting of the Association for
  Computational Linguistics}, 6 2014.

\bibitem{JMLR:v15:srivastava14a}
Nitish Srivastava, Geoffrey Hinton, Alex Krizhevsky, Ilya Sutskever, and Ruslan
  Salakhutdinov.
 Dropout: A simple way to prevent neural networks from overfitting.
 {\em Journal of Machine Learning Research}, 15:1929--1958, 2014.

\bibitem{DBLP:journals/corr/KingmaB14}
Diederik~P. Kingma and Jimmy Ba.
 Adam: {A} method for stochastic optimization.
 {\em CoRR}, abs/1412.6980, 2014.

\bibitem{boididou2015verifying}
Christina Boididou, Katerina Andreadou, Symeon Papadopoulos, Duc-Tien
  Dang-Nguyen, Giulia Boato, Michael Riegler, Yiannis Kompatsiaris, et~al.
 Verifying multimedia use at mediaeval 2015.
 In {\em MediaEval}, 2015.

\bibitem{google_2013}
Tomas Mikolov and Ilya Sutskever.
 Google code archive - long-term storage for google code project
  hosting., 7 2013.

\bibitem{Vosoughi:2016:TLT:2911451.2914762}
Soroush Vosoughi, Prashanth Vijayaraghavan, and Deb Roy.
 Tweet2vec: Learning tweet embeddings using character-level cnn-lstm
  encoder-decoder.
 In {\em Proceedings of the 39th International ACM SIGIR Conference on
  Research and Development in Information Retrieval}, SIGIR '16, pages
  1041--1044, New York, NY, USA, 2016. ACM.

\end{thebibliography}


\end{document}
