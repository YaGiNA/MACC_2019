\section{研究目的}
\label{ch:purpose}
%
今回対象とするカテゴリの投稿例を今回扱ったデータセットから抜粋したものが以下の図\ref{fig:examples}である.

\begin{figure*}[ht]
    \centering
    \begin{subfigure}[b]{0.4\textwidth}
        \includegraphics[height=6.5cm]{images/real_example_boston.jpg}
        \caption{Boston RIC released this flier showing at large suspect Dzhokhar Tsarnaev. He may be armed \& dangerous}
        \label{fig:real}
    \end{subfigure}
    \hfill % separation between the subfigures
    \begin{subfigure}[b]{0.57\textwidth}
        \includegraphics[height=6.5cm]{images/humor_example_boston.jpg}
        \caption{The detail of these photos used to identify the Boston Marathon bombing suspect is bananas…}
        \label{fig:humor}
    \end{subfigure}
    \bigskip 
    \centering
    \begin{subfigure}[b]{\textwidth}
        \includegraphics[width=\textwidth]{images/fake_example_boston.jpg}
        \caption{Reddit is on to something... Boston Bomber \#2 sure looks like missing student Sunil Tripathi. }
        \label{fig:fake}
    \end{subfigure}
    \caption{当研究で扱う3カテゴリの投稿例: (a)正しいニュース,(b)ジョークニュース,(c)フェイクニュース}
    \label{fig:examples}
\end{figure*}

いずれも2013年に発生したボストンマラソン爆弾テロ事件に関してTwitter上で投稿されたものであった.
図\ref{fig:real}は実際にボストン市傘下が作成した被疑者の情報をChicago Sun-TimesがTwitterに投稿したもの,
図\ref{fig:humor}はテロ後にRedditや4chanにて有志が実行犯の調査が行われた件に対して
``bananas''と茶化すような言葉を投げかけているもの,
図\ref{fig:fake}は実際に上記掲示板上で実行犯の調査が行われた結果,
全くの別人を槍玉に挙げているものである.

実際に,ボストンマラソン後ではインターネット上で盛んに犯人探しが行われた結果,
事件前に行方不明になっていたスニル・トリパティ(Sunil Tripathi)さんが犯人として扱われ,
更にその後一般報道メディアもトリパティさんの家族に取材が行われるなど,
フェイクニュースが実害として現実になった\cite{gray_2013}.
これを受け,Redditでは実際に犯人探しの過熱で無関係の個人とその家族に迷惑をかけたとして謝罪した\cite{laird_2013}.

%
当研究では,上記対象を正確に3カテゴリへ分類するモデルを構築することを目標としている.
具体的には,入力として画像と文章を持ち,それに対してどのカテゴリが該当するかを出力するモデルとなる.

当研究を更に発展させると,SNS上でユーザや運営側を支援するエージェント開発に繋げることができる.
%