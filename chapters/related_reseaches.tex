\section{関連研究}
\subsection{画像・文章の分析}
画像分類のは近年目まぐるしい発展を遂げた.
特に画像の被写体から分類するタスクにおいては,
VGG19のように16-19層の畳み込み層(CNN)を取り入れたモデルが非常に高い分類成果を挙げることが報告
\cite{DBLP:journals/corr/SimonyanZ14a}された.
また,VGG19を含めた多くのモデルでは,事前学習済みモデルが配布されているため,自分で転移学習を行うことも容易である.
文章に関しても,画像と同じく並列実行が可能なCNNをテキスト用にアレンジしたテキストCNNも提案
\cite{DBLP:journals/corr/Kim14f}され,広く使われている.

画像と文章を組み合わせた研究も数多くなされてきた.
例えば,画像をCNNで分析してLSTMによってキャプションを生成する研究\cite{7298935}によって,
より精度の高いキャプション生成ができたことが報告された.
他にも画像に対して文章で視覚質問(画像に写ったものを問う)
に応答することを目的としたVQA\cite{7410636}というモデルも提案された.

\subsection{フェイクニュース対策}
現在,フェイクニュースを判断する手法の1つに有識者によって事実関係を確認するファクトチェックがある.
例えばPolitifact.comではTruth-o-meterという独自指標によって,
政治的主張に対して疑わしさを7段階で評価\cite{holan_2018}している.

フェイクニュース自体への対策が発展していく中で,
フェイクニュースを``真実か嘘か''という基準から判断すること自体に疑義を唱える取り組みも存在する.
現在,Mike Tamir氏によって立ち上げられたFakerFactというwebアプリケーションがある\cite{tamir}.
このwebサイトではWaltという独自のAIを搭載しており,
文章を``真実か嘘か''は判断せず,文章の論調を分類していた.

このようにフェイクニュースを判断するにあたって,
近年では``真実か嘘か''という観点にとらわれない多くのアプローチや分類が行われていることがわかる.
