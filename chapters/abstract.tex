\begin{abstract}
    %背景
    SNSの発展によりあらゆる情報入手が容易になった反面、故意に作成された虚偽の情報であるフェイクニュースが社会問題になっている。
    特に画像と併せて発信されたものは、テキストのみならず画像と併せた分析アプローチが有効である。

    %既存課題
    既にテキスト・画像をCNNによって分析して真偽を判定する自動判別モデルが提案されているものの、
    ジョークとしての嘘情報と人を欺くための嘘情報が区別されていない。

    %提案
    本研究では、正しい情報・ジョーク情報・人を欺くための情報の3カテゴリを分類することで、より画像つきフェイクニュースの検出精度を向上させることを目指した.


    %実験結果
    実験結果、3カテゴリでもマクロF値が0.93と良好な結果を示した。
\end{abstract}
%